% Options for packages loaded elsewhere
\PassOptionsToPackage{unicode}{hyperref}
\PassOptionsToPackage{hyphens}{url}
%
\documentclass[
]{article}
\usepackage{amsmath,amssymb}
\usepackage{lmodern}
\usepackage{iftex}
\ifPDFTeX
  \usepackage[T1]{fontenc}
  \usepackage[utf8]{inputenc}
  \usepackage{textcomp} % provide euro and other symbols
\else % if luatex or xetex
  \usepackage{unicode-math}
  \defaultfontfeatures{Scale=MatchLowercase}
  \defaultfontfeatures[\rmfamily]{Ligatures=TeX,Scale=1}
\fi
% Use upquote if available, for straight quotes in verbatim environments
\IfFileExists{upquote.sty}{\usepackage{upquote}}{}
\IfFileExists{microtype.sty}{% use microtype if available
  \usepackage[]{microtype}
  \UseMicrotypeSet[protrusion]{basicmath} % disable protrusion for tt fonts
}{}
\makeatletter
\@ifundefined{KOMAClassName}{% if non-KOMA class
  \IfFileExists{parskip.sty}{%
    \usepackage{parskip}
  }{% else
    \setlength{\parindent}{0pt}
    \setlength{\parskip}{6pt plus 2pt minus 1pt}}
}{% if KOMA class
  \KOMAoptions{parskip=half}}
\makeatother
\usepackage{xcolor}
\usepackage[margin=1in]{geometry}
\usepackage{longtable,booktabs,array}
\usepackage{calc} % for calculating minipage widths
% Correct order of tables after \paragraph or \subparagraph
\usepackage{etoolbox}
\makeatletter
\patchcmd\longtable{\par}{\if@noskipsec\mbox{}\fi\par}{}{}
\makeatother
% Allow footnotes in longtable head/foot
\IfFileExists{footnotehyper.sty}{\usepackage{footnotehyper}}{\usepackage{footnote}}
\makesavenoteenv{longtable}
\usepackage{graphicx}
\makeatletter
\def\maxwidth{\ifdim\Gin@nat@width>\linewidth\linewidth\else\Gin@nat@width\fi}
\def\maxheight{\ifdim\Gin@nat@height>\textheight\textheight\else\Gin@nat@height\fi}
\makeatother
% Scale images if necessary, so that they will not overflow the page
% margins by default, and it is still possible to overwrite the defaults
% using explicit options in \includegraphics[width, height, ...]{}
\setkeys{Gin}{width=\maxwidth,height=\maxheight,keepaspectratio}
% Set default figure placement to htbp
\makeatletter
\def\fps@figure{htbp}
\makeatother
\setlength{\emergencystretch}{3em} % prevent overfull lines
\providecommand{\tightlist}{%
  \setlength{\itemsep}{0pt}\setlength{\parskip}{0pt}}
\setcounter{secnumdepth}{-\maxdimen} % remove section numbering
\ifLuaTeX
  \usepackage{selnolig}  % disable illegal ligatures
\fi
\IfFileExists{bookmark.sty}{\usepackage{bookmark}}{\usepackage{hyperref}}
\IfFileExists{xurl.sty}{\usepackage{xurl}}{} % add URL line breaks if available
\urlstyle{same} % disable monospaced font for URLs
\hypersetup{
  pdftitle={Proyecto Modelación Estadística en Ecología},
  hidelinks,
  pdfcreator={LaTeX via pandoc}}

\title{Proyecto Modelación Estadística en Ecología}
\author{}
\date{\vspace{-2.5em}}

\begin{document}
\maketitle

Estos datos se obtuvieron a partir del estudio de Ueno \emph{et al.}
2015
\href{https://doi.org/10.1111/1365-2435.12519}{DOI:10.1111/1365-2435.12519},
en el cual los autores infirieron el rendimiento de \emph{Lolium
multiflorum} a partir de la cantidad de biomasa (g por planta), bajo los
efectos de la exposición a Ozono troposférico (presencia/ausencia), la
interacción del hongo endófito \emph{Epiclöe ocultans}
(presencia/ausencia) y la exposición a herbívoros (presencia/ausencia de
áfidos).

A partir de estos datos, queremos modelar la cantidad de biomasa de las
plantas en respuesta a los distintos tratamientos.

\hypertarget{diseuxf1o-experimental}{%
\section{Diseño experimental}\label{diseuxf1o-experimental}}

\includegraphics{diseño.png}

\hypertarget{variables}{%
\section{Variables}\label{variables}}

\begin{longtable}[]{@{}llll@{}}
\toprule()
Variable & Nombre & Escala & Tipo \\
\midrule()
\endhead
Respuesta & Biomasa & (g/planta) & Continua \\
Efecto & Año & NA & Categórica \\
Efecto & Ozono & (presencia/ausencia) & Categórica Binaria \\
Efecto & Endófitos & (presencia/ausencia) & Categórica Binaria \\
Efecto & Áfidos & (presencia/ausencia) & Categórica Binaria \\
\bottomrule()
\end{longtable}

\hypertarget{datos}{%
\section{Datos}\label{datos}}

\begin{longtable}[]{@{}lllrrr@{}}
\toprule()
Ozono & Endófitos & Herbivoría & Indviduos & Media & SD \\
\midrule()
\endhead
Ozono & Endófitos & Sin Áfidos & 12 & 6.815000 & 2.785807 \\
Ozono & Endófitos & Áfidos & 12 & 4.978333 & 3.021826 \\
Ozono & Sin Endófitos & Sin Áfidos & 12 & 5.322500 & 2.725423 \\
Ozono & Sin Endófitos & Áfidos & 12 & 3.445833 & 2.248674 \\
Sin Ozono & Endófitos & Sin Áfidos & 12 & 7.240000 & 3.983498 \\
Sin Ozono & Endófitos & Áfidos & 12 & 5.195000 & 2.986039 \\
Sin Ozono & Sin Endófitos & Sin Áfidos & 12 & 5.239167 & 2.514613 \\
Sin Ozono & Sin Endófitos & Áfidos & 12 & 4.349167 & 3.684013 \\
\bottomrule()
\end{longtable}

\hypertarget{distribuciuxf3n-de-la-variable-de-respuesta}{%
\section{Distribución de la variable de
respuesta}\label{distribuciuxf3n-de-la-variable-de-respuesta}}

\includegraphics{ProyectoModelado_files/figure-latex/unnamed-chunk-5-1.pdf}

\hypertarget{distribuciuxf3n-de-los-datos-por-variables}{%
\section{Distribución de los datos por
variables}\label{distribuciuxf3n-de-los-datos-por-variables}}

\includegraphics{ProyectoModelado_files/figure-latex/unnamed-chunk-6-1.pdf}

\hypertarget{hipuxf3tesis}{%
\section{Hipótesis}\label{hipuxf3tesis}}

El desempeño de las plantas puede ser inferido a partir de su biomasa.
Esta característica puede verse afectada por factores abióticos, como la
exposición a distintos estrese ambientales; y bióticos como la
interacción con organismos mutualistas o antagonistas. Se ha reportado
que la caracteristica mutualista de los hongos endófitos se ve afectada
por el contexto ambiental pudiendo llegar a tener un comportamiento
antagonista en codiciones ambientales adversas. Teniendo esto en cuenta,
se plantean las siguientes hipótesis:

\begin{itemize}
\tightlist
\item
  El desempeño de la planta se verá afectado en condiciones ambientales
  adversas y en presencia de herbívoros.
\item
  Los hongos endófitos ayudan a su planta hospedera a sobrellevar
  condiciones ambientales adversas, por lo que el desempeño será mayor
  en presencia de endófitos.
\item
  Los hongos endófitos ayudan a su planta hospedara a sobrellevar el
  estrés por herbivoría.
\item
  En condiciones ambientales adversas los hongos endófitos cambian su
  estatus mutualista, por lo que incrementan el efecto del daño por
  herbivoría y condiciones ambientales adversas.
\end{itemize}

Con estas hipóstesis biológicas se puede plantear los siguientes modelos
estadísticos, considerando el año en que se realizó el experimento como
un efecto aleatorio.

\[
\tag{Eq. 1}
y_B= \beta_{0} + \beta_{1}x_{o} + \beta_{2} x_{e} + \beta_{3}x_{a}    
\]

En Eq. 1 se está planteando que todas las variables de efecto tienen
efecto sobre el rendimiento, expresado en producción de biomasa (B). En
este caso el modelo únicamente incluye el intercepto y un conjunto de
vcvariables \emph{dummy}. Siendo \emph{o} la concentración de ozono a la
cual se expusieron las plantas, valiendo 1 si la planta se expuso a
concentraciones alta de ozono y 0 si únicamente se expuso al ozono
ambiental. La inoculación con endófitos está representada con \emph{e},
valiendo 1 si se incoluaron los individuos con hongos endófitos y 0 si
no. Por último, \emph{a} se refiere a si los individuos se expusieron a
áfidos (\emph{a} = 1) o no (\emph{a} = 0).

Teniendo esto en cuenta se pueden desarrollar los siguientes modelos:

\[
\tag{Eq. 2}
y_B= \beta_{0} + \beta_{1}x_{o}   
\] En Eq. 2 se plantea que la diferencia en la producción de la biomasa
está dada por los cambios en la concentración de ozono. Por otro lado la
hipótesis relacionada con la presencia o ausencia de asociados
mutualistas, se puede expresar con el siguiente modelo

\[
\tag{Eq. 3}
y_B= \beta_{0} + \beta_{2}x_{e} 
\] Continuando con los modelos más simples, la hipótesis del efecto de
la presencia de herbívoros sobre el rendimiento de la planta se
representaría con el siguiente modelo

\[
\tag{Eq. 4}
y_B= \beta_{0} + \beta_{3}x_{a}
\]

El modelo planteado para la hipótesis relacionada con el cambio en la
característica mutualista de los endófitos con respecto a las
concentraciones de ozono es

\[
\tag{Eq. 5}
y_B= \beta_{0} + \beta_{1}x_{o} + \beta_{2}x_{e} + \beta_{4}x_{e}x_{o}
\] en este modelo se decidió incluir la interacción entre el ozono y la
presencia de endófitos.Para el efecto de los herbívoros en diferentes
concentraciones de ozono sobre el desempeño de la planta se planteo

\[
\tag{Eq. 6}
y_B= \beta_{0} + \beta_{1}x_{o} + \beta_{3}x_{a} + \beta_{5}x_{a}x_{o}
\]

Por último la última hipótesis presentada relaciona las tres variables
de efecto presentadas, este modelo se diferencía de Eq. 1 porque incluye
las ihnteracciones entre las variables

\[
\tag{Eq. 7}
B= \beta_{0} + U_{0}+\beta_{1}x_{o} + U_{1}x_{o}+\beta_{2} x_{e}+U_{2} x_{e} + \beta_{3}x_{a} +U_{3}x_{a}  + \beta_{4}x_{e}x_{o}  +U_{4}x_{e}x_{o}  + \beta_{5}x_{a}x_{o} +  U_{5}x_{a}x_{o} +\beta_{6}x_{a}x_{e} + U_{6}x_{a}x_{e} + \beta_{7}x_{a}x_{e}x_{o} + U_{7}x_{a}x_{e}x_{o} 
\]

\end{document}
